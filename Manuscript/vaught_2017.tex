\documentclass[twocolumn]{aastex61}
\usepackage{natbib}
\bibliographystyle{apj}
\usepackage{graphicx}
\usepackage{amsmath}

\newcommand{\citeth}[1]{(\citeauthor{#1}\ \citeyear{#1})}
\newcommand{\citethnop}[1]{\citeauthor{#1}\ \citeyear{#1}}
\newcommand{\citethnopp}[1]{\citeauthor{#1}\ (\citeyear{#1})}



\def \lya {Ly$\alpha$ }
\def \mkms {{\rm \; km\;s^{-1}}}
\def \msunperyr {{\; M_{\odot}\rm \;yr^{-1}}}

\begin{document}
\title{A VLT/FORS2 Narrowband Imaging Search for \ion{Mg}{2} Emission Around $\lowercase{z}\sim0.7$ Galaxies }
\author{Ryan Rickards Vaught}
\affiliation{Department of Astronomy, San Diego State University, San Diego, CA 92182}
\affiliation{Center for Astrophysics and Space Sciences, Department of Physics, University of California, San Diego, CA} 
 
 \author{Kate H. R. Rubin }
 \affiliation{Department of Astronomy, San Diego State University, San Diego, CA 92182}
 
 \author{Fabrizio Arrigoni Battaia }
 \affiliation{European Southern Observatory, Karl-Schwarzschild-Str. 2, D-85748 Garching bei M\"unchen, Germany}

 \author{J. Xavier Prochaska}
 \affiliation{Astronomy \& Astrophysics, UC Santa Cruz, 1156 High St., Santa Cruz, CA 95064 USA}
 
\author{Joseph F. Hennawi}
\affiliation{Department of Physics, Broida Hall, University of California, Santa Barbara, CA 93106-9530}

\correspondingauthor{Ryan Rickards Vaught}
\email{rjrickar@ucsd.edu}

\begin{abstract}

The mass and energy of galactic winds remain poorly constrained via traditional absorption line studies. One way to better constrain these parameters is to measure the spatial extent of the outflow in emission. We perform a VLT/FORS2 narrowband imaging of 5 star-forming galaxies at redshift $z=0.67-0.69$ in the GOODS-S field as part of an effort to spatially resolve and constrain the radial extent of large-scale outflows traced by \ion{Mg}{2} emission. These observations probe extended \ion{Mg}{2} emission down to unmatched surface brightness limits of 5.74 $\times$ $10^{-19}$ ergs sec $^{-1}$ cm$^{-2}$ arcsec$^2$ (5$\sigma$).  We do not detect any extended \ion{Mg}{2} emission in any of our galaxies, thus placing strong upper limits on the strength of \ion{Mg}{2} emission at projected distances $R_{\perp} = 13-24$ kpc from the sample galaxies. Our observations allow us to create the first ever spatially-resolved map of \ion{Mg}{2} absorption, revealing approximately constant absorption strengths across the galaxy disks. Our detection limits in concert with a previous studies Keck/LRIS spectra allow us to model \ion{Mg}{2} emission for isotropic and anisotropic dust-free winds. We find that anisotropic winds reduce the strength of estimated missing \ion{Mg}{2} emission below our 5$\sigma$ limit.

\end{abstract}

\keywords{galaxies: evolution, galaxies: absorption lines}

\section{Introduction}\label{sec:intro}
Galactic winds play a critical role in regulating the star formation rates and stellar masses of galaxies \citeth{Werk_2014}; however, the physical mechanism that powers these galactic winds remains uncertain. Some possible mechanics have been proposed by theoretical studies that include thermal pressure from core collapse supernova, radiation pressure from starbursts and finally cosmic ray pressure (\citethnop{Larson_1974}; \citethnop{Chevalier_1985}; \citethnop{Springel_2003}; \citethnop{Sugahara_2017}). Additionally, the impact galactic winds have on their host galaxies (i.e, their mass and energy content) has remained difficult to constrain with observations.

An accurate picture of what types of galaxies host outflows comes from numerous absorption line studies of galaxies (\citethnop{Veilleux2005}; \citethnop{Weiner2009}; \citethnop{Martin2012}; \citethnop{Rubin_2014}). Outflows are detected by measuring the blueshift (outflow) or redshift (inflow) of absorption transitions with respect to the host galaxy systematic velocity.
 %Spectroscopy of galaxies from low to high redshifts probing cold gas ($T \lesssim 10^2$ K) which absorbs in \ion{Na}{1} and cool gas ($T \sim 10^4$ K) in \ion{Mg}{2} has revealed outflows in most galaxies that host active star-formation (e.g. \citethnop{chen2010}; \citethnop{Martin2012}; \citethnop{Rubin_2014}).
Even though these probes can constrain: the radial velocity, column density and covering fraction of the flow, 
this method is weak in constraining the overall radial extent and provides no information on the morphology of the outflow.

One way of constraining the radial extent and morphology of the outflow is to trace winds in emission. This has been demonstrated using rest-frame optical transitions (i.e H$\alpha$, [\ion{O}{3}], etc) as tracers for winds around nearby starbursts (\citethnop{Matsubayashi2009}; \citethnop{Veilleux2009}; \citethnop{Tripp2011}) as these transitions can trace the warm shock heated phase of the gas. 
A novel, alternative transition potentially useful for tracing winds in emission is the \ion{Mg}{2} $\lambda\lambda 2976,2803$ doublet in the rest-frame ultraviolet (UV)
(Weiner et al. 2009; Kornei et al. 2013)\nocite{Weiner2009, Kornei2013}. While most studies of winds using \ion{Mg}{2} have focused on its absorption kinematics, \cite{Rubin_2011} observed  strong \ion{Mg}{2} emission with a P-Cygni line profile in the spectrum of a strongly star-forming galaxy at redshift $z = 0.694$.  This emission was the first detection of an outflow beyond the spatial extent of the galaxy continuum in the distant universe, with a minimum spatial extent  $>$ 7 kpc.

One proposed production mechanism for the galaxy's observed \ion{Mg}{2} line profile is photon scattering. In this mechanism, \ion{Mg}{2} ions in the region of the wind closest to the observer will absorb continuum photons in the resonant transitions, $\lambda_{2796}$ and $\lambda_{2803}$. Once these transitions are excited, they may only decay back to the ground state and if the optical depth of the gas is high, then the gas will resonantly trap the absorbed photons. Because the photons are absorbed in the rest frame of the gas, the absorption will be blueshifted relative to the galaxy's systematic velocity. The \ion{Mg}{2} ions in the section of the wind farthest from the observer will absorb and scatter photons that are redshifted relative to the front portion of the wind. Because the photons are redshifted, the photons will be able to travel freely towards the observer through the wind to produce emission at and redward of the systematic velocity of the galaxy (\citethnop{Rubin_2011}, \citethnop{Prochaska_2011}). Since the first detection of \ion{Mg}{2} emission by \citet{Rubin_2011}, another detection was reported by \cite{Martin2013}, who observed \ion{Mg}{2} emission that extends $12-18$ kpc from a strongly star-forming galaxy
at $z=0.9392$.   \ion{Mg}{2} has also been studied in wide-field galaxy surveys conducted with Keck/DEIMOS and VLT/MUSE  (\citethnop{Erb2012}, \citethnop{Feltre2018}). These surveys, which include galaxies with redshifts $ 0.70 < z < 2.30$, find that \ion{Mg}{2} may be detected in pure emission, pure absorption or with P-Cygni profiles.  Detections of \ion{Mg}{2} in emission were found to be associated with galaxies of lower stellar mass and bluer spectral slopes.

The diversity of these spectral profiles may be understood using radiative transfer modeling of galactic winds.
%Various models of galactic winds are explored in 
\citet{Prochaska_2011} have used this technique to predict 
%This analysis involved generating 
spectra for the  \ion{Mg}{2} and  \ion{Fe}{2}$^*$ transitions for a variety of wind morphologies.
%, using a set of radiative transfer algorithms, for an isotropic and anisotropic set of dust and dust-free winds. 
The authors demonstrated that isotropic dust-free winds will conserve photon flux, thus predicting that blueshifted absorption lines should be accompanied by emission profiles of winds with similar equivalent widths (EW). Anisotropic winds, however, were demonstrated to exhibit significantly weaker emission at strengths proportional to the angular extent (i.e., solid angle) of the wind. The scattered emission was additionally weakened by the inclusion of dust and the presence of a strongly-absorbing interstellar medium (ISM).
Thus, spatially-resolved measurement of the surface brightness of this emission constrains not only the radial extent of the emitting material, but also its morphology and dust content.

In this paper, we present the first narrowband imaging of the \ion{Mg}{2} transition around 5 star-forming  galaxies located in the Great Observatories Origins Deep Survey south (GOODS-S) field at redshift $z \sim 0.7$. 
%The value in using narrowband imaging is the ability to spatially resolve any absorption and emission from a galaxy. 
We use two filters: a line filter and a continuum filter. The line filter covers the \ion{Mg}{2} doublet, while the continuum filter is offset from the line filter by ${\sim}47$ \AA.    
The resulting imaging in each filter has a total integration time of 10 hrs.
%The continuum filter allows us to observe the continuum level of the galaxy near the \ion{Mg}{2} doublet. With the resulting images taken using the two filters, we can subtract the continuum image from the \ion{Mg}{2} image, effectively creating an image that contains only photons from \ion{Mg}{2}. 
As opposed to 1D spectra, the narrowband imaging constrains the surface brightness, optical depth and radial extent of the wind. This narrowband imaging allows us to create the first ever high-S/N spatially resolved map of \ion{Mg}{2} emission and absorption. 

In Section \ref{sec:obs_red} we describe our sample of GOOD-S galaxies, supplemental Keck/LRIS spectra, as well as our VLT/FORS2 observations, image reduction, and absolute flux calibration. We go into detail in our method of continuum subtraction in Section \ref{sec.cont_sub}. The analysis of these data is presented in Section \ref{sec:analysis}, 
including our methods for calculating surface brightness profiles and detection limits for each galaxy, as well as maps of \ion{Mg}{2} equivalent widths.
%in detail we discuss the production surface bright profiles and limits for each galaxy, as well as the production of narrowband %measured equivalent widths. 
Section \ref{sec:results} presents results from our narrowband imaging analysis. We compare our SB detection limits to previous detections of extended \ion{Mg}{2} emission,  and compare our observations to predictions made by radiative transfer models in Section \ref{sec:discussion}. We conclude this paper in Section \ref{sec:conclusion}.
We adopt a $\Lambda$CDM cosmology with $h_{70} = H_0/(70\ \rm{ km}\ \rm{ s}^{-1}\ \rm{ Mpc}^{-1})$, $\Omega_{\rm{M}}= 0.3$, and $\Omega_{\Lambda} = 0.7$. In this cosmology, 1\arcsec is $\approx 7\ \rm kpc$ at $z \sim 0.7$.


\section{Observations and Data Reduction}\label{sec:obs_red}
\subsection{Sample Selection}
Our target galaxies were selected from a Keck/LRIS survey of UV absorption lines in $\approx 100$ objects having redshifts $0.3< z < 1.4$ and $B_{AB}< 23$ in fields with deep \emph{HST}/ACS imaging \citep{Rubin_2014}.  In particular, this parent survey targeted galaxies in a total of nine Keck/LRIS pointings located in both of the GOODS fields (\citeauthor{Giavalisco2004} \citeyear{Giavalisco2004}) and the AEGIS survey field (the Extended Groth Strip; Davis et al. 2007)\nocite{Davis2007}.  In inspecting the redshift distribution of the portion of this sample observable from the Southern Hemisphere, we uncovered a narrow peak of nine galaxies in the interval $0.66 < z < 0.68$.  This peak is in fact the global maximum of the distribution, as all other bins of width $\Delta z = 0.02$ have at most four galaxies.  Moreover, there are two narrow interference filters available on VLT/FORS2 centered at $\lambda \sim 4675$ and 4722 \AA\ which cover the \ion{Mg}{2} $\lambda \lambda 2796, 2803$ transition in precisely this redshift interval.  We selected our final sample of five of these galaxies at $0.66 < z < 0.68$ to be close on the sky such that they could be imaged in a single $7' \times 7' $ FORS2 pointing.  
We show color {\it HST}/ACS images of these objects in Figure~\ref{fig:hstims}.

The absorption line modeling presented in \cite{Rubin_2014} indicates that these five galaxies are driving strong outflows traced by \ion{Mg}{2}  with velocities $\sim150-420\mkms$ and equivalent widths (EW) $\sim 2-3$ \AA.  Modeling of the galaxy broad-band spectral energy distributions obtained from multi-wavelength ancillary imaging data yields star formation rates (SFR) ranging from $\sim4$ to $40\msunperyr$ and stellar masses in the range $\log M_*/M_{\odot}\sim 9.9-11.0$. The properties and nomenclature of the sample, as well as precise target coordinates taken from \cite{Rubin_2014} ) are listed in Table \ref{tab:prop}. 


\subsection{VLT/FORS2 Observations}
Our narrow-band imaging data were taken in service-mode using the FORS2 instrument on the VLT 8.2m telescope Antu between October 2012 and February 2013. 
We used two narrowband filters, HeII+47 and HeII/3000+48, that have peak transmission at wavelengths that correspond to the \ion{Mg}{2} doublet lines at our sample redshift of $z\sim0.7$ (see Table \ref{tab:filters}). The filter transmission curves are plotted along with each galaxy's spectrum in Figure~\ref{fig:spec_images}.
In the following, we will often refer to the HeII+47 filter as the ``line'' or \ion{Mg}{2} filter and the HeII/3000+48 filter as the ``continuum'' filter.

FORS2 has a native pixel scale of $0.125''$ pixel$^{-1}$ and a field of view of $7'\times7'$.  The data were taken with 
the CCD binned $2\times2$, yielding a pixel scale of $0.25''$ pixel$^{-1}$.
Images of three pointings offset by $0.25\arcmin$ East/West were obtained, with individual exposure times of $\approx$ 1000 sec.  A total of 38  exposures were taken in each filter. 
Our observations were carried out under photometric and thin cloud conditions (program ID: 090.A-0427A). The seeing, which ranged from $0.5\arcsec$ to $2.5\arcsec$, are given in the image header and are derived from DIMM seeing measurements. These DIMM values are calculated from zenith observations of bright stars at 0.5 microns, which are then reduced to the zenith angles and wavelengths of the science observation. Additionally, a first order correction is then applied for appropriate for a 8m telescope is applied (see Figure \ref{fig.seeing})  Summing the individual exposure times for each filter results in a combined exposure time of $10.0$ hours each for the HeII+47 and HeII/3000+48 filters.

\begin{figure*}[!ht]
\centering
\includegraphics[scale=.73]{../Figures/fors2_color_imstamps.pdf}
\caption{Color imaging of our sample galaxies in the \emph{HST}/ACS F435W, F606W, and F775W filters obtained as part of the GOODS survey (\citeauthor{Giavalisco2004} \citeyear{Giavalisco2004}). Each image is 5'' $\times$ 5'' ( or about $ 40 \rm \ kpc \times 40 \rm \ kpc $) .\label{fig:hstims}}
\end{figure*}

\begin{figure*}[!h]
\centering
\gridline{\fig{../Figures/filt_26_spectra.pdf}{0.5\textwidth}{(a)}
          \fig{../Figures/filt_36_spectra.pdf}{0.5\textwidth}{(b)}}
\gridline{\fig{../Figures/filt_03_spectra.pdf}{0.5\textwidth}{(c)}
          \fig{../Figures/filt_64_spectra.pdf}{0.5\textwidth}{(d)}}
           \fig{../Figures/filt_57_spectra.pdf}{0.5\textwidth}{(e)}
\caption{Keck/LRIS spectra of the sample galaxies and the transmission curves of the filters HeII+47 (blue dashed line) and HeII/3000+48 (red dashed line) . The left-hand axis is in units of flux density and the right-hand axis is the percentage of light transmitted by the filter at each wavelength. Vertical dashed lines indicate the wavelengths of the redshifted \ion{Mg}{2}\ doublet. The \ion{Mg}{2}\ doublet falls fortuitously at the central wavelength of the HeII+47 filter for the galaxies shown in panels (a) through (d).}
\label{fig:spec_images}
\end{figure*}

\begin{figure}[h]
\centering
\includegraphics[scale=.55]{../Figures/avg_seeing_HEII.pdf}
\includegraphics[scale=.55]{../Figures/avg_seeing_HEII3000.pdf}
\caption{\textbf{Top}: Distribution of seeings for the 38 HeII+47 images.
\textbf{Bottom}: Same HeII/3000+48 images. The median seeing conditions for the images in both filters is $\sim 0.8\arcsec$. 
The seeing conditions were calculated from DIMM measurements which are provided by ESO in the header of each science image.
\label{fig.seeing}}
\end{figure}

\subsection{Supplemental Keck/LRIS spectra}
In addition to VLT imaging, in the present analysis we utilize galaxy spectra taken from the \cite{Rubin_2014} Keck I Low Resolution Imaging Spectrometer (LRIS) program.  A $0.9''$ slit width was used for all slitmasks and the median FWHM resolution for the spectra is 274 km s$^{-1}$ at $\lambda_{\rm rest} \approx$ 2800 \AA\ and $286\mkms$  at $\lambda_{\rm rest}\approx$ 2600 \AA\ (see Figure \ref{fig:spec_images}).  The spectral coverage of these data extends from ${\sim}3200$ to 8000 \AA.

\subsection{Image Reduction}
The data were reduced using custom routines written in \textbf{Python}. 
The images were first corrected by subtracting and removing the overscan region of the CCD. 
Then the images were bias-subtracted and flat-fielded using twilight flats.
To improve the flat-fielding essential for detecting faint extended emission across the fields,we further correct for the illumination patterns using night-sky flats. The night-sky flats were produced by combining the unregistered science frames with an average sigma-clipping algorithm after masking out all the objects, and any bad pixels. Each individual image was cleaned of cosmic rays and bad pixels by utilizing the \emph{L.A. Cosmic} algorithm \citep{Dokkum2001}.
The astrometry solutions were calculated via Astrometry.net (Lang et al. 2010)\nocite{Lang}, and yield a standard deviation in the galaxy coordinates of $\sigma \approx 0.10''$. Before image stacking, we ran each frame through \emph{SExtractor} \citeth{Bertin} in order to create a root mean square (RMS) map of each science image.


The final stacked image for each filter is obtained using \emph{SWarp} \citeth{Bertin}.
Each individual frame is first sky-subtracted using a background mesh size of 256 pixels which is approximately $64''$. 
We chose the mesh size to be large enough such that the extended emission is not mistakenly subtracted (Arrigoni Battaia et al. 2015)\nocite{Battaia_2015}. 
The frames, after background-subtraction, are resampled onto a common astrometric solution using a \textit{Lancosz3} interpolation kernel. 
The images are weighted by the night-flat image and then  average-combined  in order to increase the signal-to-noise of any \ion{Mg}{2} emission. Additionally, \emph{SWarp} generates stacked RMS images by propagating the error images for each science frame.
%weighting and average-combining the RMS images for individual science frames.
 %from the night flats and the combination of RMS images for individual science frames. 
Our final stacked images in each filter are shown in Figure~\ref{fig:stacked_image} with the target galaxies indicated.  

\begin{table*}[t]
\centering
\caption{Properties of the 5 galaxies in our sample as estimated in \cite{Rubin_2014}. The EW (in the observed frame) include both components of the \ion{Mg}{2} doublet and were 
determined from analysis of the supplemental Keck/LRIS spectra.}
\begin{tabular}{llllllll} \hline \hline
Object\footnote{ Name in parenthesis to be used as shorthand for the objects throughout the remainder of the paper } & R.A. & Dec  & $z$ & SFR($M_{\odot}$ yr$^{-1}$) & $\log{M_{*}/M_{\odot}}$ & EW$_{\rm{obs}}$(\AA) & $\tau_V$\footnote{ 97.5th percentile of the $\tau_V$ distribution}\smallskip      \\ \hline 
J033225.26-274524.0 (J.26)      & 03:32:25.26 & -27:45:23.9 & 0.6660 & $9.1_{-3.7}^{+1.3}$& $9.86_{-0.04}^{+0.05}$ & $7.539\pm 0.354$ & 4.812\\ 

J033231.36-274725.0 (J.36)      & 03:32:31.35 & -27:47:24.9 &   0.6669 & $10.5_{-1.6}^{+1.7}$ & $10.02_{-0.03}^{+0.03}$&$5.835 \pm 0.493$ & 3.427\\

J033230.03-274347.3  (J.03)     & 03:32:30.03 & -27:43:47.2  &   0.6679 & $3.8_{-0}^{+0}$ & $10.98_{-0.0}^{+0.01}$ &$12.794 \pm 1.710$ & 2.867 \\

J033229.64-274242.6  (J.64)    & 03:32:29.64 & -27:42:42.5 & 0.6671 & $40.5_{-12.1}^{+8.2}$ & $10.30_{-0.03}^{+0.07}$ &$13.239 \pm 0.263$ & 5.042\\

J033230.57-274518.2  (J.57)    & 03:32:30.56 & -27:45:18.2 &   0.6807  & $12.6_{-2.1}^{+1.7}$ & $10.48_{-0.07}^{+0.03}$ &$6.106 \pm 0.370$ &4.022 \\

\hline 
\end{tabular}
\label{tab:prop}
\end{table*}

\begin{table}[h!]
\caption{Filter properties  and exposure times of the VLT/FORS2 observations. The width of the transmission curves ($\Delta\lambda$) are calculated by convolving the transmission curve with the total wavelength range of the filter. These values differ slightly from those reported by the European Space Observatory. }
\begin{tabular}{llllll} \hline \hline 
Filter & $\lambda_{\rm{eff}}$(\AA)\footnote{$\lambda_{\rm{eff}}$ is the effective wavelength of the filter transmission curve.} & $\Delta\lambda$(\text{\AA})    & $N$\footnote{Total number of images}   & $T(s)$\footnote{Total exposure time} & $S$\footnote{S, the sensitivity of the filter, is in units of $10^{-17}$ ergs counts$^{-1}$ cm$^{-2}$ }\smallskip  \\ \hline 
HeII+47  & 4675.21 & 50.11 & 38  & 35,959 & 2.45 \\
HeII/3000+48 & 4722.46  & 44.82 & 38 &   36,937 & 2.40  \\ \hline
\end{tabular}
\label{tab:filters}
\end{table}


\subsection{Absolute Flux Calibration}
We acquired observations of the standard star GD50 from archival ESO calibration imaging at 4 independent epochs. Performing aperture photometry at each epoch and airmass, we calculated the atmospheric extinction coefficients ($k$) to be 0.181 magnitudes for the HeII/3000+48 filter and 0.190 magnitudes for the HeII+47 filter. We perform absolute flux calibration using the methods of Jacoby et al. (1990). We first convolve the spectral energy distribution of the standard star, $F(\lambda)$ in ergs sec$^{-1}$ \AA$^{-1}$ cm$^{-2}$, with that of the known transmission curve of the filter, $T_{i}(\lambda)$. This yields $F_i$, the total observable flux in each bandpass filter $i$ with units of ergs sec$^{-1}$ cm$^{-2}$:
\begin{equation*}
F_{i}=\int F(\lambda)T_{i}(\lambda)d\lambda.
\end{equation*}
It is not uncommon to assume that $F(\lambda)$ is constant over the small width of the filter. 
%However, our filter transmission curves are sampled at $5$\ \AA\ intervals which is the same sampling as the spectral energy distribution of the standard star obtained from the ESO archives (WILL CHECK CALSPEC). 
However, we calculate the integral numerically.
The system sensitivity, including the defects of the telescope optics and detector response is then given by
\begin{equation*}
S_{i}=\dfrac{F_{i}}{C10^{k_{i}A}},
\end{equation*}\\
where $k_i$ is the extinction in magnitudes per airmass, A is the airmass for each individual exposure, C is the measured count rate of the standard star and $S_i$ is in units of ergs counts$^{-1}$ cm$^{-2}$. Before image co-addition, each science image is corrected for atmospheric extinction by multiplying each frame by $10^{k_{i}A}$. Next, the image is divided by the exposure time, effectively putting the image in units of counts per sec. After co-addition, the images are then multiplied by the appropriate sensitivity factor $S_{i}$. This puts the final images in the appropriate flux units, ergs sec $^{-1}$ cm$^{-2}$.

\begin{figure*}[ht!]
\centering
\includegraphics[scale=.61]{../Figures/HEII_final.png}
\includegraphics[scale=.61]{../Figures/HEII3000_final.png}
\caption{Top: Stacked HeII+47 image of the galaxy sample. Bottom: Stacked HeII3000+48 image of the same pointing. The exposure time of each image is $\approx 10$ hours. Each image shows half of the total FOV of $7' \times 7'$ which contains the full sample of galaxies (indicated by the white arrows). East is up and North is right.
\label{fig:stacked_image}}
\end{figure*}

\section{Image Subtraction}\label{sec.cont_sub}
We have two goals for our study: (1) assess the surface brightness of line emission in the \ion{Mg}{2} transition in and around each target galaxy; and (2) spatially resolve the morphology of the strong \ion{Mg}{2} absorption observed against the galaxy continua.
To reach both of these goals, we must perform accurate subtraction of the continuum flux of each object from the filter covering the targeted line emission. For four of the five galaxies in our sample, the HeII+47 image includes both line and continuum emission, and the HeII3000+48 image provides a high S/N measurement of the continuum only $\approx30$ \AA\ redward of the line emission in the rest-frame. The galaxy excluded from our analysis is instead used as a check on the quality of our continuum subtraction for reasons detailed in Section \ref{subsubsec:SBprofiles}.

\subsection{Spectral Correction}
In preparation for continuum subtraction, we first consider whether the continuum level of each galaxy spectrum changes significantly over the passbands of our two filters.
We use the supplementary spectra from \citet{Rubin_2014} to fit the continuum and determine the spectral slope of each galaxy. We use the interactive fitting routine \emph{lt\_continuumfit} from the \emph{linetools} package (Prochaska et al. 2016)\footnote{https://github.com/linetools/linetools}\nocite{Prochaska2016} to fit the continuum. We then find the total continuum flux in each filter by convolving the fitted continuum with each filter's transmission curve. Next, we take the ratio of both integrated totals, as the ratio will indicate the scaling factor needed to correct our flux measurements prior to continuum subtraction. Comparing these ratios between each galaxy, we find that they are effectively the same, within 0.1\%, with a value of 1.118. This value is equal to the ratio between the FWHM of the filter transmission curves, allowing us to conclude that the slope of the spectrum of each galaxy is flat, and hence that the continuum level measured in the off-line filter provides an accurate measure of the continuum contribution to the on-line filter flux.

\subsection{Continuum Subtraction}\label{subsec.cont_sub}

To properly continuum-subtract the image taken with the \ion{Mg}{2} filter, we follow a prescription given by \cite{Battaia_2015}. 
We first determine the continuum flux density from the continuum filter,
\begin{equation}
f_{\rm{cont}}=\frac{F_{\rm{cont}}}{\Delta \lambda_{\rm{cont}}},
\end{equation}\\
where $F_{\rm{cont}}$ and $\Delta \lambda_{\rm{cont}}$ are the observed flux per pixel of the continuum image and the transmission FWHM of the continuum filter, respectively. With $f_{\rm cont}$ it is then possible to calculate the flux of any excess emission, $F_{\rm{line}}$:
\begin{equation}
F_{\rm{line}}=F_{\rm{MgII}}-f_{\rm{cont}} \Delta \lambda_{\rm{MgII}}
\label{eq:subtraction}
\end{equation}
where $F_{\rm{MgII}}$ and $\Delta \lambda_{\rm MgII}$ are the observed flux per pixel in the \ion{Mg}{2} filter and the transmission FWHM of the \ion{Mg}{2} filter. The continuum subtracted images of each galaxy are shown in Figure \ref{fig:stamp_images}. The continuum subtracted image has uniform background and no obvious signatures of emission.

\begin{figure*}[!htb]
\centering
\includegraphics[scale=0.7]{../Figures/stamps.pdf}
\caption{ $10'' \times 10''$ (or about $70 \rm \ kpc \times 70 \rm \ kpc $) images of each galaxy in our sample. Top row: Continuum surface brightness in ergs s$^{-1}$ cm$^{-2}$ arcsec$^{-2}$ measured in the HeII/3000+48 filter. Bottom row: Continuum-subtracted \ion{Mg}{2} surface brightness.  Absorption can be seen in 4 of 5 galaxies. The red contours represent the outline of the 1$\sigma$ surface brightness limit in the HeII+47 image, defined in Sec. \ref{sec.sb}. The colorbar shows the scaling used for the \ion{Mg}{2} images.}
\label{fig:stamp_images}
\end{figure*}

\section{Analysis} \label{sec:analysis}
%   We detail below the resulting surface brightness profiles (Section~\ref{sec.sb}) and detection limits (Section \ref{subsec:test}) as well at the EW$_{\rm{MgII}}$ generated from these data (Section~\ref{subsec.ew}).

\subsection{Surface Brightness Profiles and Limits}\label{sec.sb}

In order to test for the presence of \ion{Mg}{2} emission, we perform aperture photometry on the continuum subtracted images using the python library \emph{Photutils}. We choose annuli with a radial thickness of 1 pixel or $0.25 ''$, such that, $r_{inner}=r_{outer}-1$ (in pixels). Each annulus is centered on the flux-weighted centroid of the galaxy. By dividing the summed flux in each annulus by the area in arcseconds we produce surface brightness (SB) profiles for each galaxy. These profiles are shown in Figure \ref{fig:sb_profiles}. 

\begin{figure*}
\centering
\gridline{\fig{../Figures/J_26.pdf}{0.5\textwidth}{(a)}
          \fig{../Figures/J_36.pdf}{0.5\textwidth}{(b)}}
\gridline{\fig{../Figures/J_03.pdf}{0.5\textwidth}{(c)}
          \fig{../Figures/J_64.pdf}{0.5\textwidth}{(d)}}
           \fig{../Figures/J_57.pdf}{0.5\textwidth}{(e)}
\caption{SB profiles for our sample galaxies. (Top panel) Continuum SB profile (black) measured for each galaxy. The green points show the Mg II + continuum SB measured for the galaxy in the pre-continuum subtracted image. The blue points show the Mg II line SB measured for the galaxy in the continuum subtracted line emission image.  The profile exhibits SB decrements from \ion{Mg}{2} absorption. Photometry was performed in circular annuli. (Bottom panel) The vertical hashes in the bottom show the inner and outer radius of each annulus in kpc. Distance from the center of the galaxy (x-axis) is computed using the average value of the inner and outer radii of each annulus.}
\label{fig:sb_profiles}
\end{figure*}

The error in the SB is determined from the stacked RMS images of each object.  We adopt annuli that are identical to the annuli used to find the SB profiles for each galaxy. To calculate the variance inside each annulus, we sum the RMS pixel values in quadrature, then divide by the area of each annulus. 

To calculate the $1\sigma$ SB limit we follow the procedure of \cite{Battaia_2015}. We first mask out all the sources, their associated extended halos, and edge noise in both the HeII+47 and HeII/3000+48 images. We then calculate the RMS of the background in randomly-placed $1\arcsec$ apertures. We convert these RMS values to SB limits per $1~\rm arcsec^2$  aperture. We find that the 1$\sigma$ detection limits %per 1 arcsec$^2$ aperture 
(SB$_1$) are $6.332\times10^{-19}$ ergs sec $^{-1}$ cm$^{-2}$ arcsec$^2$ and $5.808\times10^{-19} $ ergs sec $^{-1}$ cm$^{-2}$ arcsec$^2$ in the HeII/3000+48 and HeII+47 filters, respectively. With the 1$\sigma$ detection limit, SB$_1$, determined for the continuum+\ion{Mg}{2} (HeII+47) image, we define a thicker (or ``extended") annulus to be used to search for any extended \ion{Mg}{2} emission. This extended annulus has an inner radius approximately the size of the SB$_1$ isophotal contour for each galaxy. The outer radius is chosen to be the inner radius plus 5 pixels. Measurements of \ion{Mg}{2} SB will probe projected distances of 13, 18, 21 and 24 kpc from the centers of each target. The resulting SB measurements are shown in Figure~\ref{fig:sb_profiles}.

The sensitivity required to detect an extended source depends on its size, as one can reach lower SB levels by spatially averaging over large apertures. In the ideal case of perfect sky subtraction and continuum subtraction, the 1$\sigma$ SB limit for an extended source is $SB_{1}/\sqrt{A_\text{src}}$, where $A_\text{src}$ is the area in arcsec$^2$ and $SB_{1}$ is the surface brightness limit per 1 arcsec$^2$ aperture. In practice, the actual detection limits are affected by systematics from imperfect subtraction. Therefore, we empirically determine the limits as follows. We mask all the artifacts and sources in the continuum-subtracted images. Next, we generate apertures with sizes similar to our extended annulus $\sim 20 \rm\ sq.arcsec$, place them at random, and extract the fluxes, $F_{\text{src}}$, within these apertures.

 In the ideal case that the sky and continuum are perfectly subtracted the value of $F_{\text{src}}/ \sigma_{\text{src}}$, where $\sigma_{\text{src}} \equiv SB_{1}\sqrt{A_\text{src}}$, from many random apertures should follow a Gaussian distribution with unit variance. The distribution of $F_{\text{src}}/\sigma_{\text{src}}$ for these apertures is shown in Figure \ref{fig:limits}. We calculate the variance and mean of the distribution and find that the variance of the distribution is $\sigma'_{\text{src}}=1.1$ and implies that the SB detection limit for our continuum-subtracted image is broader than $\sigma_{\text{src}}$ by 10\%. We thus adopt $F_{\text{limit}} \equiv \sigma'_{\text{src}}$  as the $1 \sigma$ upper limit on the total line flux of extended \ion{Mg}{2} emission. The SB$_{\text{limit}}$ is then $F_{\text{limit}}/A_{\rm{src}}$. The values of $F_{\text{src}}$ and the 5SB$_{\text{limit}}$ for each galaxy are listed in Table \ref{tab:det_lims}.

\begin{figure}[!ht]
\centering
\includegraphics[scale=0.6]{../Figures/hist_sblim.pdf}
\caption{Normalized distribution of $F_{\text{src}}/\sigma_{\text{src}}$ values for random circular annuli placed on the continuum-subtracted line image. $F_{\text{src}}$ is the total flux within an aperture and $\sigma_{\text{src}}$ is the expected $1\sigma$ flux limit in the ideal case of perfect sky and continuum subtraction, i.e $SB_{1}\sqrt{A_\text{src}}$. The black arrows point to the statistical significance of the flux inside the ``extended annulus'' of each galaxy.}
\label{fig:limits}
\end{figure}

\begin{table}[h]
\centering
%\begin{minipage}{4in}
\caption{Significance of extracted flux and detection limits\label{tab:det_lims}}  
\begin{tabular}{lllll} \hline \hline
Object & F$_{\rm{src}}$(\ion{Mg}{2})\footnote{\ion{Mg}{2} flux is in $10^{-18}$ ergs sec$^{-1}$ cm$^{-2}$. The value in the parenthesis is the statistical significance with respect to $\sigma_{\text{src}}$ } & 5SB$_{\text{limit}}$\footnote{Limits are in $10^{-19}$ ergs sec $^{-1}$ cm$^{-2}$ arcsec$^{-2}$} & Area\footnote{Area of the extended annulus in sq.arcsec} \\  \hline
J033225.26-274524.0 &  2.44(0.92)& 6.51	& 21 \\
J033232.36-274725.0 &  -1.40(-0.53)& 6.51 & 21 \\
J033230.03-274347.3 &  -5.23(-1.75)& 5.74 & 27 \\
J033229.64-274242.5 &  1.23 (0.44) & 6.22  &26 \\
J033230.57-274518.2 &  -2.53 (-1.00) & 6.81 &18 \\ \hline
\end{tabular}
%\end{minipage}
\end{table}

\subsection{Test of Surface Brightness Limits}\label{subsec:test}
In order to show that our detection limits are reasonable, we simulate emission with varying intensities. The results of this exercise are shown in Figure \ref{fig:sigmas}. For each galaxy, we assign our simulated emission a constant surface brightness corresponding to 1, 3, 5, 10 and 20 times the $1\sigma$ SB$_{\text{limit}}$ inside the largest annulus used (i.e., the extended annulus). Additionally, we assume Gaussian noise with 1$\sigma$ equal to 1SB$_{\text{limit}}$. After placing the simulated emission around the galaxy, we subtract the continuum in the same manner as explained in Section \ref{subsec.cont_sub}. 

We then construct a smoothed $\chi$ image following the techniques in \cite{Hennawi2013} and \cite{Battaia_2015}. First, we smooth the continuum-subtracted image:

\begin{equation}
I_{\text{smooth}}= \text{CONVOLVE[NB-CONTINUUM]},
\end{equation}
where the CONVOLVE operation indicates convolution of the images with a Gaussian kernel with FWHM=1.5 pixels. Next, we computed the sigma image ($\sigma_{\text{smooth}}$) for the smoothed image ($I_{\text{smooth}}$) by propagating the noise image of the unsmoothed data:
\begin{equation}
\sigma_{\text{smooth}}=\sqrt{\text{CONVOLVE}^2[\sigma^2_{\text{unsmooth}}]},
\end{equation}
where the CONVOLVE$^2$ operation indicates the convolution of the variance image with the square of the Gaussian kernel. The smoothed $\chi$ image is defined by
\begin{equation}
\chi_{\text{smooth}}=\frac{I_{\text{smooth}}}{\sigma_{\text{smooth}}}.
\end{equation}
This $\chi_{\text{smooth}}$ image aids in recognizing the presence of extended \ion{Mg}{2} emission. 

Figure \ref{fig:sigmas} shows the $\chi_{\text{smooth}}$ images for the 5 levels of simulated \ion{Mg}{2} emission. We also include the $\chi_{\text{smooth}}$ image of each galaxy without any simulated emission (in the left most column). The galaxies are outlined by a black isophotal contour corresponding to 1SB$_1$ and the simulated emission is contained inside the extended annulus surrounding each contour. The  $\chi_{\text{smooth}}$ images confirm that we should be able to detect extended \ion{Mg}{2} emission down to a conservative level of 5SB$_{\text{limit}}$. Note again that the SB$_{\text{limit}}$ does indeed take into account the systematics from imperfect continuum subtraction. 


\begin{figure*}[p]
\centering
\includegraphics[scale=1.2]{../Figures/sigmas.pdf}
\caption{Postage stamp $\chi_{\text{smooth}}$ images for the Mg II emission of the 5 galaxies in our sample. Every galaxy is placed in the same row in each column. The columns show simulated emission, with levels of 0, 1, 3, 5, 10, and 20 times SB$_{lim}$.  Each postage stamp has a size of $5'' \times 5''$ (corresponding to $35$ kpc $\times$ $35$ kpc at $z\sim 0.70$). Each stamp shows the galaxy along with the same isophotal contour used in previous figures.}
\label{fig:sigmas}
\end{figure*}


\subsection{Equivalent Widths}\label{subsec.ew}
Here we derive an expression to calculate the equivalent width (EW$_{\rm{MgII}}$) of any absorption or emission features observed in our narrow-band imaging. Starting from the expression for EW used in the context of spectroscopy,
\begin{equation}
EW_{\lambda}=\int (1-\frac{f_{\lambda}}{f_{\rm{cont}}})d\lambda
\label{eq:specEW}
\end{equation}
we begin by dividing Eq \ref{eq:subtraction} by the flux density of the continuum and the FWHM of the on-line filter,
\begin{equation}
\frac{F_{\lambda}}{f_{\rm{cont}}\Delta \lambda_{\rm{MgII}}}=\frac{F_{\rm{MgII}}}{f_{\rm{cont}}\Delta \lambda_{\rm{MgII}}}- 1.
\end{equation}
Next, we rearrange the above expression such that we produce the argument of the integrand in Eq. \ref{eq:specEW} on the right hand side,
\begin{equation}
-\frac{F_{\lambda}}{f_{\rm{cont}}\Delta \lambda_{\rm{MgII}}}=1-\frac{f_{\rm{MgII}}}{f_{\rm{cont}}}.
\end{equation}
We then approximate the integration in Eq. \ref{eq:specEW} by multiplying the integrand above by the FWHM of the on-line filter $d\lambda=\Delta \lambda_{\rm{MgII}},$
\begin{equation}
-\frac{F_{\lambda}}{f_{\rm{cont}}}=(1-\frac{f_{\rm{MgII}}}{f_{\rm{cont}}})\Delta \lambda_{\rm{MgII}};
\end{equation}
such that
\begin{equation}
EW_{\rm{MgII}}=-\frac{F_{\rm{MgII}}}{f_{\rm{cont}}}.
\end{equation}

Using the above equation along with the continuum and continuum-subtracted images, we produce images of the observed-frame EW$_{\rm{MgII}}$. The observed-frame EW$_{\rm{MgII}}$ images are displayed for each galaxy in Figure \ref{fig:ews} and show only the EWs within the 1$\sigma$ SB$_1$ contours of the corresponding \ion{Mg}{2} images (prior to continuum subtraction). 

To compare our map of EW$_{\rm{MgII}}$ to the values measured from the Keck/LRIS spectra, we place 0.9 arcsec wide apertures on top of each galaxy. The width and angle of the apertures replicate the orientation of the slits used to obtain the spectra. Next, we determine which pixels lie outside the 1$\sigma$ SB$_1$ contours and set their values to zero. Outside this contour, the EW$_{\rm{MgII}}$ values become poorly constrained due to the lack of S/N in the continuum. We then apply a S/N $= 1.5$ cut to the continuum image and create a histogram to show the distribution of EW$_{\rm{MgII}}$ inside the contour. The histograms are shown in Figures \ref{fig:ew11} through \ref{fig:ew55}. Having removed pixels with low-S/N continuum values from our images, we compute the mean equivalent width inside the Keck/LRIS apertures. These values are summarized in Table \ref{tab:abs_props}.
 
To assess the morphology of the \ion{Mg}{2} absorption, we determine the distance of each pixel from the center of each galaxy in kiloparsecs. We plot the EW$_{\rm{MgII}}$ over this projected distance for each galaxy in Figures \ref{fig:ew11} through \ref{fig:ew55}. Although some of the plots suggest a slight upward trend in the values of the EW$_{\rm{MgII}}$ with increasing radii, we cannot be confident in this trend because of the large scatter. To better visualize the data and test the significance of the trend, we bin the data radially in 3-5 kpc-wide bins. For example, in Figure \ref{fig:ew44}, the EW$_{\rm{MgII}}$ from absorption of J.03 extends out to $\sim$ 25 kpc. The EW$_{\rm{MgII}}$ values are binned in 5 kpc increments. We calculate the mean and scatter of the EW$_{\rm{MgII}}$ in each bin and show these values in Figure \ref{fig:ew_comb}.   

\section{Results}\label{sec:results}

\subsection{Spatially Resolved Maps of \ion{Mg}{2} Absorption}
In this section we discuss the details of the absorption detected in our SB profiles as well as compare our EW$_{\rm{MgII}}$ measurements to those measured in the supplemental Keck/LRIS spectra. 

\subsubsection{\ion{Mg}{2}\ absorption in surface brightness profiles} \label{subsubsec:SBprofiles}
Although we do not detect any extended \ion{Mg}{2} emission, we do observe a decrement of flux in the SB profiles of 4 out of 5 galaxies in our sample. Figure \ref{fig:sb_profiles} shows the SB profiles for the 5 galaxies in our sample. Absorption from \ion{Mg}{2} ions is prevalent in the profiles between 10 - 25 kpc, decreasing radially outward from the maximum absorption at the center of the galaxies. In Table \ref{tab:abs_props} we report for the galaxies J.26, J.36 and J.03 a maximum decrement in the SB profile due to absorption $\sim$ -5.00$\times10^{-18}$ erg s$^{-1}$ cm$^{-2}$ per sq.arcsec. Additionally, we report for galaxy J.64 a decrement one order of magnitude greater than the previous galaxies with a value of -18.2 $\pm$ 0.127 $\times10^{-18}$ erg s$^{-1}$ cm$^{-2}$ per sq.arcsec. Finally, for galaxy J.57 the absorption decrement is strongest at the center of the galaxy with a SB value of -1.25 $\pm$ 1.15 $10^{-18}$ erg s$^{-1}$ cm$^{-2}$ per sq.arcsec. However, the measured SB of this decrement including error is consistent with measuring zero absorption. The measurement of zero absorption for J.57 was expected. Figure \ref{fig:spec_images} shows the Keck/LRIS spectrum of J.57 and the transmission curves of the filters HeII+47 and HeII/3000+48. The \ion{Mg}{2} transitions of the galaxy are equally sampled by both of our filters. When we subtract the continuum image from the \ion{Mg}{2} image we are effectively subtracting the \ion{Mg}{2} as well. This measurement suggests that the quality of our subtraction is satisfactory.  


The surface brightness profiles presented in Figures \ref{fig:sb_profiles} do not exhibit any signs of extended \ion{Mg}{2} emission.

\subsubsection{Morphology of MgII Absorption}
Figures \ref{fig:ews}  show the images, distributions and radial projections of \ion{Mg}{2} EWs. We have zeroed out any values that lie outside the SB$_1$ contours for each galaxy. We also impose a signal-to-noise cut, only including EWs where the continuum S/N is greater than $1.5$. The mean EW is computed for all pixels inside each Keck/LRIS aperture, defined in Sec. \ref{subsec.ew}, and the resulting values are summarized in Table \ref{tab:abs_props}. Comparing our narrowband EWs with those measured from the spectra, we find agreement to within a 1.6-4.6$\sigma$ for galaxies J.26 and J.36, and more statistically significant differences for galaxies J.03 and J.64 We discuss possible causes for these differences below.

Figure \ref{fig:spec_images}(c) shows the Keck/LRIS spectrum of galaxy J.03. The continuum observed near the \ion{Mg}{2} transition is noisy, compared to the spectra of the rest of the sample. Since the value of the EW is dependent on the level of the continuum present near the \ion{Mg}{2} transition, it may well be that our choice of continuum level in calculating the EW from the spectrum might be higher relative to the continuum level implied by our narrow-band image.
%might be over-estimating the EW of the doublet.

Figure \ref{fig:spec_images}(d) shows the Keck/LRIS spectrum of galaxy J.64. This galaxy is the brightest galaxy in the sample, and also exhibits the highest-velocity wind.  This shifts the \ion{Mg}{2} absorption profile toward the blue end of the 
HeII+47 transmission curve, which could cause the flux in this filter to be weighted toward the continuum level and the absorption signal to be underestimated.

%suggestive of a bright ISM, which causes the redder transition to almost leak out of the HeII+47 transmission curve. This suggests %that it our \ion{Mg}{2} images are not measuring the most redward absorption feature, thus underestimating the value of the EW.

%In the case of J.57 we can easily understand why the EW measured is not in agreement with the spectral EW. Figure \ref{fig:sb_profiles}  shows the SB profile coverage of J.57. In this galaxy's profile we do not detect any significant SB decrements, due to absorption, as the two filters cover the \ion{Mg}{2} as discussed above. Any signal of absorption in the EW images arises from noise. Figure \ref{fig:ew55} shows that the radial projection of the EWs measured for this galaxy that are centered around zero.

%To get a better sense of how the EW of \ion{Mg}{2} absorption changes with radius as well as account for the effects of seeing, %$FWHM \sim 7 kpc$, we bin the EWs in radial bins with 5, 4 or 3 kpc widths as shown in Figures \label{fig:ew1} through %\label{fig:ew5}. 
As demonstrated in our radially-binned EW profiles, a majority of the galaxies exhibit large scatter in the binned EWs out to larger radii. To better understand the significance of these trends, we construct a plot that compiles the mean EWs for all the galaxies, shown in Figure \ref{fig:ew_comb}. To take into account the varying size of the galaxies, we normalize the distances by the approximate radius of the SB$_1$ contour for each galaxy. 

Upon inspection of Figure \ref{fig:ew_comb}, we see that the galaxies exhibit no statistically significant trend in the mean absorption EW as a function of radius inside our 1SB$_1$ isophotal contour, which suggests that the covering fraction is approximately constant across the surface.

\subsection{Limits on \ion{Mg}{2} Emission}
For our sample of galaxies, we are sensitive to extended emission at minimum projected distances of 
13, 18, 21 and 24 kpc, respectively, from the centers of each target.
%outside the center of the galxies at distances 8-21 kpc fr
We do not detect any significant \ion{Mg}{2} emission at these distances around any of our target galaxies. The $\chi_{\text{smooth}}$ images shown in Figure \ref{fig:sigmas} confirm this. A comparison of the simulated emission with the $\chi_{\text{smooth}}$ image of the original continuum-subtracted image, shown in the first column, similarly suggests that we do not detect any extended \ion{Mg}{2} emission. We thus place $5\sigma$ upper limits on \ion{Mg}{2} emission for each galaxy in the sample, summarized in Table \ref{tab:det_lims}. The most sensitive detection limit using the largest area is SB(\ion{Mg}{2}) = 5.74 $\times$ $10^{-19}$ ergs sec $^{-1}$ cm$^{-2}$ arcsec$^2$, computed for the galaxy J.03. 
\section{Discussion}\label{sec:discussion}

\subsection{Previous Detections of Extended \ion{Mg}{2} Emission}
Previous constraints on the brightness of scattered \ion{Mg}{2} emission were reported by \cite{Rubin_2011}. In this work the authors studied emission from the starburst galaxy TKRS 4389 at $z = 0.69$ with a star formation rate of $49.8\msunperyr$. This emission was detected in a 2-dimensional Keck/LRIS spectrum, with flux from the emission reaching $(8.0 \pm 0.4)$ and $(4.4 \pm 0.4)$ $\times10^{-18}$ ergs sec$^{-1}$ cm$^{-2}$ at  $\lambda _{2796}$ and $(4.0 \pm 0.3)$ and $(2.5 \pm 0.4)$ $\times10^{-18}$ ergs sec$^{-1}$ cm$^{-2}$ at $\lambda_{2803}$ in two independent locations spatially offset from the galaxy continuum. The flux from the emission can be converted into two surface brightness values by taking the average of the flux measured at each location and each transition, and dividing by a 1 sq.arcsec aperture. 

Figure \ref{fig:SFR_lim} shows a plot of the 5$\sigma$ detection limits determined for each galaxy in our sample as well as the SB calculated for the galaxy TKRS 4389 vs SFR. The figure suggests that we should be able to detect scattered \ion{Mg}{2} emission with strengths similar to that detected in TKRS 4389. Taken at face value the figure could be consistent with a positive correlation between \ion{Mg}{2} SB and SFR. Future observations are needed to verify this trend, and thus the possibility of detect \ion{Mg}{2} in emission for high SFR, $>\sim$50 $\msunperyr$. Figure \ref{fig:detection_lim} shows the same SB values as Figure \ref{fig:SFR_lim} vs $\log(M_*/M_{\odot})$. The figure is suggestive that \ion{Mg}{2} is stronger in emission in lower mass galaxies, and is consistent with the results in \cite{Erb2012} and \cite{Feltre2018}.

\begin{figure}[!htb]
\centering
\includegraphics[scale=0.6]{../Figures/limits.pdf}
\caption{Comparison of our detection limits to the measured extended emission of TKRS 4389 vs SFR. The arrow color-galaxy scheme is: (Blue) J.26, (Green) J.36, (Red) J.03, (Cyan) J.64, (Magenta) J.57. Our imaging is sufficiently sensitive to detect extended emission at similar strengths to the extended emission measured for TKRS 4389 with SFR $\sim 50 \msunperyr$.}
\label{fig:SFR_lim}
\end{figure}

\begin{figure}[!htb]
\centering
\includegraphics[scale=0.6]{../Figures/mass_limits.pdf}
\caption{Comparison of our detection limits to the measured extended emission of TKRS 4389 vs Mass. Our imaging is sufficiently sensitive to detect extended emission at similar strengths to the extended emission measured for TKRS 4389 with mass $\sim 10 \log(M_*/ M_{\odot})$. The color scheme is the same as Figure \ref{fig:SFR_lim}.}
\label{fig:detection_lim}
\end{figure}

\subsection{Geometry of Scattering Material}
In idealized wind models of cool gas outflows, discussed in \cite{Prochaska_2011}, radiative transfer calculations suggest that strong \ion{Mg}{2} emission is predicted to be generated along with ubiquitous blueshifted absorption of \ion{Mg}{2}. For isotropic and dust-free scenarios, photons are conserved, as any absorbed continuum photon is eventually re-emitted, most likely at a different rest wavelength.  Therefore, the total equivalent width of both the absorption and emission features is equal to zero in such models. Assuming that our galaxies host an isotropic and dust-free wind, we wish to determine how much emission is predicted to be generated by this wind, and how the SB of this emission compares to our detection limits.

To calculate the predicted emission flux we first determine the flux absorbed by \ion{Mg}{2} ions. Using our Keck/LRIS spectra, we find the average value of the continuum near the \ion{Mg}{2} doublet and multiply this value by the observed EW of the doublet. Then to estimate the SB, we distribute this flux uniformly inside multiple annuli of varying sizes. These annuli all have an inner radius equal to the galaxy's isophotal radius and successively larger outer radii.  Additionally, since our SB limits are dependent on the size of the aperture used, we calculate the SB detection limits of our images inside each of the aforementioned annuli. Figure \ref{fig.emission} shows how the SB of emission varies with the spatial extent of the annulus (black squares), as well as how the SB compares with our detection limits (thin black curve). Excepting the galaxy J.03, the SB of this ``missing'' emission lies above our detection limits. %Without taking into account that the emission may not exceed to a radial extent outside the galaxy disk, 
Under the assumption that the wind in these galaxies does in fact extend beyond the $\rm SB_1$ isophotal contour (at $R_{\perp} = 8-21$ kpc), this result suggests that these galaxies do not host  isotropic, dust-free winds.  

There are many phenomena that may reduce the SB of the scattered \ion{Mg}{2} emission so that it is consistent with our observations. Introducing dust into the wind can reduce the observed emission strength, and affect the shape of the line profile. 
%In the absense of dust, the back scattered \ion{Mg}{2} photon can escape the galaxy free of collisions. With dust, the photons are %susceptible to a number of collisions, with a greater possibility of being absorbed by dust. 
In the \cite{Prochaska_2011} radiative models that include dust in the wind material, the dominant effect %that dust has on the
 is that  the most redshifted emission is suppressed. The strength of the flux is reduced by a factor of $(1+\tau_{\rm{dust}})^{-1}$, where $\tau_{\rm{dust}}$ is the integrated opacity of dust.  [Check on dust content implied by SED modeling from Rubin+14.]

Another factor that can reduce the amount of emission is anisotropy of the wind. Direct evidence of anisotropic winds, specifically bipolar morphology, has been observed in emission from cold and shock heated gas in local starburst galaxies \citep[e.g.][]{{Walter2002,Westmoquette2008M,Strickland2009}}. For distant galaxies, enhanced absorption \ion{Mg}{2} along a galaxy's minor axis (Bordoloi et al. 2011; Bouche? et al. 2012; Kacprzak et al. 2012),  higher outflow velocities toward $z \sim 1$ galaxies having $i < 45 ^{\circ}$ are suggestive of bipolar outflows. Furthermore, \cite{Rubin_2014} analysis provide strong evidence for the ubiquity of bipolar outflows in distant galaxies.

We now assume that the emission in our galaxies is reduced by the effect of anisotropy. 
For the anisotropic winds modeled in \cite{Prochaska_2011}, the emission is reduced by the factor $\Omega/4\pi$, where $\Omega$ is the angular extent of the wind. 
As \citet{Prochaska_2011} points out, given that the outflow must cover most of the continuum in order to be detected, the value of $\Omega$ has an approximate lower limit of $\Omega > 2\pi$. 
We show the predicted SB profiles for wind emission from our galaxies assuming  $\Omega = 2\pi$ with blue diamonds in Figure  \ref{fig.emission}.

After reducing the SB of the expected \ion{Mg}{2} emission by the corresponding factor of 2, we predict profiles that fall below our SB detection limits for galaxies J.26 and J.36. However, the SB profile of J.64 remains above our SB detection limits, suggesting additional phenomena are needed to reduce the strength of scattered emission. Previously discussed in Section \ref{sec:results}, this object is the intrinsically brightest galaxy of our sample and exhibits the strong \ion{Mg}{2} absorption, which suggests a strong ISM component. \cite{Prochaska_2011} note that \ion{Mg}{2} photons can be more effectively trapped in objects with a large amount of dusty interstellar material.


\begin{table*}[]
\centering
\caption{Properties of \ion{Mg}{2}\ Absorption\label{tab:abs_props}}  
\begin{tabular}{llllll} \hline \hline
Object & Max(SB$_{abs}$)\footnote{SB Values are in units of $10^{-18}$ ergs sec $^{-1}$ cm$^{-2}$ arcsec$^2$} & $R_{\text{SB}_1}(kpc)$ &\ \ \ EW$_{\rm{obs}}$($\AA$)\footnote{ Measured from EW images} & EW$_{\rm{obs}}$($\AA$)\footnote{Measured from Keck/LRIS spectra}  \\  \hline
J033225.26-274524.0 &  $-5.39 \pm 1.28 $ & 8 &     $\ \ \ 3.511 \pm 0.795$ & $7.539 \pm 0.354 $\\
J033232.36-274725.0 &  $-5.64 \pm 0.121 $ & 15 & $\ \ \ 7.694 \pm 1.016$ & $5.835 \pm 0.493$\\
J033230.03-274347.3 &  $-4.41 \pm 0.622 $ & 21 & $\ \ \ 5.388 \pm 0.712$ & $12.79 \pm 1.710$\\
J033229.64-274242.5 &  $-18.2 \pm 0.127 $ & 10 & $\ \ \ 7.546 \pm 0.660$ & $13.24 \pm 0.263$\\
J033230.57-274518.2 &  $-1.25 \pm 1.15   $ & 11& -$0.6532 \pm 0.589$ & $6.106 \pm 0.370$\\ \hline
\end{tabular}
\end{table*}


\section{CONCLUSION}\label{sec:conclusion}
We presented the results from a narrowband imaging search for \ion{Mg}{2} emission in a sample of star-forming galaxies at a redshift of $z \sim 0.70$ which are known to drive outflows in  \ion{Mg}{2}. We did not detect any \ion{Mg}{2} emission in this sample, and place a $5\sigma$ detection upper limit of SB(\ion{Mg}{2}) $\approx$ 5.74 $\times 10^{-19}$ ergs sec $^{-1}$ cm$^{-2}$ arcsec$^2$, measured in an extended annulus outside the largest galaxy, at minimum projected distances of 13, 18, 21 and 24 kpc from the centers of each target. We detected a significant amount of \ion{Mg}{2} absorption in a total of 4 galaxies out of our 5 galaxy sample. Furthermore, our imaging allows us to generate spatially-resolved maps of \ion{Mg}{2} absorption in a distant galaxy sample for the first time. This absorption covers the center of the galaxies out to an isophotal radius defined by the SB$_1$ (1$\sigma$) depth of a continuum + \ion{Mg}{2} image, at approximately $\sim 22$ kpc, suggesting that the absorbing gas fully covers the stellar disks out to this distance. The EWs measured in these maps are in broad agreement with measured EWs using Keck/LRIS integrated slit spectroscopy. Additionally, our radial projections of the mean EW for our sample galaxies suggest that the EWs due to \ion{Mg}{2} are approximately constant across the galaxies surfaces. 

We compared our detection limits with the prediction of the radiative transfer models of \cite{Prochaska_2011}. We are able to rule out that the winds in our sample are isotropic and dust free, as our images are sufficiently sensitive to detect the predicted emission from such models. By assuming anisotropic winds, we were able to reduce the strength of the predicted emission to lie below our detection limit for 3 of our sample galaxies. We must invoke additional phenomena like dusty ISM material 
to reconcile the predicted SB of emission and our detection limits in the fourth sample galaxy. 


Based on the radial extent and strength of our measured \ion{Mg}{2} absorption, the \ion{Mg}{2} is optically thick. Deeper images in \ion{Mg}{2} are required to observe any extended emission beyond the depth of our images.  Although our detection limits suggest that the winds in our sample are not isotropic and dust-free, questions linger regarding the relative roles wind anisotropy, dust content, and extent play in reducing scattered emission.

A new generation of instruments such as the Keck Cosmic Web Imager (KCWI) or the Multi Unit Spectroscopic Explorer (MUSE) should be used to disentangle these issues.


\begin{figure*}
\centering
\gridline{\fig{../Figures/J26EW.pdf}{0.8\textwidth}{(a: J.26)}}
 \gridline{\fig{../Figures/J36EW.pdf}{0.8\textwidth}{(b: J.36)}}
\gridline{\fig{../Figures/J03EW.pdf}{0.8\textwidth}{(c: J.03)}}
 \gridline{\fig{../Figures/J64EW.pdf}{0.8\textwidth}{(d: J.64)}}
\caption{(Left) Stamp of the equivalent widths (EWs) inside the red 1SB$_1$ contour. The white contour represents the 0.9\arcsec aperture slit used to measure the equivalent width of absorption in the Keck/LRIS spectra. (Middle) Distribution of absorption pixels with continuum flux S/N greater than 1.5 inside the slit aperture. (Right) The EWs of absorption projected in distance from center are in blue and binned EW measurements red. The error bars in the radial axis represent the width of the radial bin used.}
\label{fig:ews}
\end{figure*}


\begin{figure*}[!htb]
\centering
\includegraphics[scale=0.9]{../Figures/ew_comb.pdf}
\caption{Normalized radial profile of the mean EW of \ion{Mg}{2} absorption for all galaxies. Measurements for the mean EW binned in radial increments compiled to test the significance of any trends found in the individual galaxy profiles. There is significant scatter relative to the error bars at extended distances.}
\label{fig:ew_comb}
\end{figure*}

\begin{figure*}[h]
\centering
\gridline{\fig{../Figures/geo_J26.pdf}{0.5\textwidth}{(J033225.26)}
          \fig{../Figures/geo_J36.pdf}{0.5\textwidth}{(J033231.36)}}
\gridline{\fig{../Figures/geo_J03.pdf}{0.5\textwidth}{(J033230.03)}
          \fig{../Figures/geo_J64.pdf}{0.5\textwidth}{(J033230.64)}}
\caption{Estimated SB profile of \ion{Mg}{2} emission for isotropic and anisotropic dust-free winds. The black square points are SB of predicted \ion{Mg}{2} emission uniformly distributed inside an annulus for an angle $\Omega=4\pi$, or an isotropic wind. The blue diamond points are SB of missing \ion{Mg}{2} emission uniformly distributed inside an annulus for an angle $\Omega=2\pi$, or an anisotropic wind. The solid line shows the value of the SB detection limits. The dashed vertical and horizontal lines represent outer radius of the extended annulus used to measure the initial detection limits and the value of the SB detection limit in that annulus.}
\label{fig.emission}
\end{figure*}

\bibliography{references2017}

\end{document}